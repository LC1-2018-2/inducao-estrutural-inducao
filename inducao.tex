\documentclass[a4paper, 10pt]{article}

\usepackage[utf8]{inputenc}
\usepackage[brazilian]{babel}

% The following packages can be found on http:\\www.ctan.org
\usepackage{graphics} % for pdf, bitmapped graphics files
\usepackage{epsfig} % for postscript graphics files
\usepackage{mathptmx} % assumes new font selection scheme installed
\usepackage{times} % assumes new font selection scheme installed
\usepackage{amsmath} % assumes amsmath package installed
\usepackage{amssymb}  % assumes amsmath package installed

\title{\LARGE \bf
 Indução e suas Aplicações
}

\author{Gabriel Teixeira da Silva}

\begin{document}
\maketitle

\begin{abstract}

Neste trabalho mostraremos diversas utilizações do princípio de indução em Ciência da Computação.

\end{abstract}

\section{Introdução}
Indução Matemática é uma técnica de demonstração usada para provar que dado um número possui certa propriedade, e é demonstrado que o sucessor desse número também a possui. Geralmente a indução é associada ao conjunto dos números naturais \textbf{N*} (1, 2, 3...). A indução Matemática pode ser empregada sempre que o conjunto em questão puder ser gerado por meios de definições indutivas.
Imaginemos que temos uma escada infinita, e queremos saber se podemos alcançar todos os degraus dessa escada.
\begin{itemize}
\item Podemos alcançar o primeiro degrau da escada.
\item Se pudermos alcançar um determinado grau da escada, então poderemos alcançar o próximo grau.
\end{itemize}
\begin{description}
\item \textbf{Exemplo}
  \hrule
\end{description}
Suponha que deseja-se provar que essa propriedade vale para todos os números do conjunto dos Naturais.
\begin{quote}
  \[1 + 2 3 +.....+ n = \frac{n(n + 1)}{2}\]
\end{quote}
\begin{itemize}
\item O primeiro passo, e mostrar que essa propriedade é válida par um 'n' qualquer.
  Esse passo é chamado de `Base de Indução(B.I).`
  \begin{center}
    \[P(n)=\frac{n(n + 1)}{2}\]

    Vamos começar essa prova com n = 1.

    \[P(1) = \frac {1(1 + 1)}{2} = 1\]
  \end{center}
\item Assumindo que a propriedade vale para um `n` qualquer, pela definição podemos provar que essa propriedade vale para o seu sucessor que é `n + 1`.Esse passo é chamado de `Passo Indutivo(P.I).`
  Devemos mostrar que n + 1 é verdadeiro, ou seja
  \begin{center}
    1 + 2 + 3 +....+ n + (n + 1) \[ = \frac{(n + 1) [(n + 1) + 1])}{2}\frac{(n +1)(n + 2)}{2}\]
    também é verdadeira. Quando adicionamos (n + 1) em ambos os lados da equação P(n), obtemos

    \[1 + 2 + 3 +.....+ (n + 1) = \frac{n(n + 1)}{2}\ + (n + 1)\]
    \[ = \frac{n(n + 1) + 2(n + 1)}{2}\]
    \[ = \frac{(n + 1) + (n + 1)}{2}\]
    \item Completamos os passos da base e indução. Assim, pela indução matemática, sabemos que P(n) é verdadeira para todos os números inteiros positivos n.

    
    \end{center}
  \end{itemize}
\section{Indução Estrutural}


\section{Conclusão}

\end{document}
