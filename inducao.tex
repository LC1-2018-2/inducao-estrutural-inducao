\documentclass[a4paper, 10pt]{article}

\usepackage[utf8]{inputenc}
\usepackage[brazilian]{babel}

% The following packages can be found on http:\\www.ctan.org
\usepackage{graphics} % for pdf, bitmapped graphics files
\usepackage{epsfig} % for postscript graphics files
\usepackage{mathptmx} % assumes new font selection scheme installed
\usepackage{times} % assumes new font selection scheme installed
\usepackage{amsmath} % assumes amsmath package installed
\usepackage{amssymb}  % assumes amsmath package installed
\usepackage{amsfonts}
\usepackage{amsthm}

\title{\LARGE \bf
 Indução e suas Aplicações
}

\author{Gabriel Teixeira da Silva}

\begin{document}
\maketitle

\begin{abstract}

Neste trabalho mostraremos diversas utilizações do princípio de indução em Ciência da Computação.

\end{abstract}

\section{Introdução}
Indução Matemática é uma técnica de demonstração usada para provar que dado um número possui certa propriedade, e é demonstrado que o sucessor desse número também a possui. Geralmente a indução é associada ao conjunto dos números naturais $\mathbb{N}$ (1, 2, 3...). A indução Matemática pode ser empregada sempre que o conjunto em questão puder ser gerado por meios de definições indutivas.
Imaginemos que temos uma escada infinita, e queremos saber se podemos alcançar todos os degraus dessa escada.
\begin{itemize}
\item Podemos alcançar o primeiro degrau da escada.
\item Se pudermos alcançar um determinado grau da escada, então poderemos alcançar o próximo grau.
\end{itemize}
\begin{description}
\item \textbf{Exemplo}
  \hrule
\end{description}
Suponha que deseja-se provar que essa propriedade vale para todos os números do conjunto dos Naturais $\mathbb{N}$.
\begin{quote}
  \[1 + 2 3 +.....+ n = \frac{n(n + 1)}{2}\]
\end{quote}
\begin{itemize}
\item O primeiro passo, e mostrar que essa propriedade é válida par um 'n' qualquer.
  Esse passo é chamado de `Base de Indução(B.I).`
  \begin{center}
    \[P(n)=\frac{n(n + 1)}{2}\]

    Vamos começar essa prova com n = 1.

    \[P(1) = \frac {1(1 + 1)}{2} = 1\]
  \end{center}
\item Assumindo que a propriedade vale para um `n` qualquer, pela definição podemos provar que essa propriedade vale para o seu sucessor que é `n + 1`.Esse passo é chamado de `Passo Indutivo(P.I).`
  Devemos mostrar que n + 1 é verdadeiro, ou seja
  \begin{center}
    1 + 2 + 3 +....+ n + (n + 1) \[ = \frac{(n + 1) [(n + 1) + 1])}{2}\frac{(n +1)(n + 2)}{2}\]
    também é verdadeira. Quando adicionamos (n + 1) em ambos os lados da equação P(n), obtemos

    \[1 + 2 + 3 +.....+ (n + 1) = \frac{n(n + 1)}{2}\ + (n + 1)\]
    \[ = \frac{n(n + 1) + 2(n + 1)}{2}\]
    \[ = \frac{(n + 1) + (n + 1)}{2}\]
    \item Completamos os passos da base e indução. Assim, pela indução matemática, sabemos que P(n) é verdadeira para todos os números inteiros positivos n.

    
    \end{center}
\end{itemize}
\hrule
\section{Indução Completa}
O passo base de uma demonstração por indução completa, é o mesmo usado na indução matemática. Para demonstrar que P(n) é verdadeira para todos os números inteiros positivos n, em que P(n) é uma função proposicional, seguimos os dois passo a seguir:
\begin{itemize}
\item \textbf{PASSO BASE} Seguindo o principio da indução matemática, o primeiro passo é verificar se P(1) é verdadeira.
\item \textbf{PASSO DE INDUÇÃO} Nesse passo vamos mostrar que a proposição condicional [P(1) /\ P(2) /\ P(3)...../\ P(N)] -> P(N + 1) é verdadeira para todos os números inteiros positivos n.
\item Dependendo do caso a ser provado, pode ser mais conveniente usar a indução matemática ou a indução completa, de uma maneira mais geral, o tipo de indução que vai ser usada, sempre vai depender do caso a ser provado.
\end{itemize}
\hrule
\newpage

\section{Indução Estrutural}
A indução estrutural pode ser entendida como uma generalização da indução matemática. A indução estrutural é usada para provar uma propriedade P para todos os elementos de um conjunto definido recursivamente. A indução matemática é fortemente ligada na estrutura recursiva dos números naturais, mas com a indução estrutural podemos apliar essa aplicação para outras estruturas definidas recursivamente.
\begin{itemize}
\item \textbf{PASSO BASE} Mostre como os resultados se mantêm para todos os elementos especificados no passo base da definição recursiva que pertencerem a um conjunto.
\item \textbf{PASSO RECURSIVO} Mostre que, se a proposição for verdadeira para cada um dos elementos usados para formar novos elementos no passo recursivo da definição, o resultado se mantém para novos elementos.
\end{itemize}
A indução estrutural pode ser usada para demonstrar que todos os elementos de um determinado conjunto construído recursivamente, têm uma propriedade particular.
\begin{quote}
  \textbf{Exemplo}
  \hrule
\end{quote}
Vamos mostrar que toda fórmula bem formada contém o mesmo número de parênteses á esquerda e á direita.
\begin{itemize}
\item Envolvem \textbf{V} e \textbf{F}
\item Variáveis proposicionais.
\item Operadores do Conjunto> ({$\neg$, $\vee$, $\wedge$, $\to$, $\leftrightarrow$}).
\end{itemize}
\textbf{Resolução}
\begin{quote}
  \textbf{Passo base}: As fórmulas \textbf{V}, \textbf{F} e p(sendo p uma variavel proposicional), não contém parênteses, portanto o mesmo número de parênteses á esquerda e á direitra são iguais (que é igual a 0).

  \textbf{Passo Indutivo}: Assuma que p e q são fórmulas bem formadas contendo o mesmo número de parênteses á esquerda e á direita. Ou seja, se l é o número de parênteses á esquerda e r e o número de parênteses á direita então: $l_{p}$ = $l_{q}$ e $r_{p}$ = $r_{q}$.
\end{quote}

  Assumindo que  $l_{p}$ = $l_{q}$ e $r_{p}$ = $r_{q}$

  Precisamos mostrar que contém o mesmo números de parênteses:
  ($\neg P$), (P$\wedge Q$), (P$\vee Q$), (P$\to Q$), (P$\leftrightarrow Q$).

  Número de parênteses á esquerda:

  Em ($\neg P$): $l_{p}$ + 1.

  Nos demais: $l_{p}$ + $l_{q}$ + 1.

  Número de parênteses á direita:

  Em ($\neg P$): $r_{p}$ + 1.

  Nos demais: $r_{p}$ + $r_{q}$ + 1.

  Uma vez que $l_{p}$ =  $r_{p}$ e $l_{q}$ =  $r_{q}$ estes números são iguais.
  \newtheorem{mydef}{Definição}
\begin{mydef}
O conjunto de árvores com raiz, em que uma árvore com raiz consistem em um conjunto de vértices que contém um vértice distinto, que chamamos de raiz, e arestas que conectam esses vértices, pode ser definido recursivamente por esses passos:
\end{mydef}
\begin{itemize}
    \item \textbf{PASSO BASE} Um único vértice r é uma árvore com raiz.
    \item \textbf{PASSO INDUTIVO} Suponha que $T_{1}$, $T_{2}$,....,$T_{n}$ são árvores disjuntas que possuem raízes $r_{1}$, $r_{2}$,....,$r_{n}$. Então, o grafo formado começando com uma raiz r, que não esteja em nenhuma das árvores com raízes $T_{1}$, $T_{2}$,....,$T_{n}$, e adicionando uma aresta a partir de r a cada um dos vértices $r_{1}$, $r_{2}$,....,$r_{n}$, é também uma árvore com raiz.
\end{itemize}
\textbf{Altura}:
\begin{itemize}
    \item A altura de um nó em uma árvore é o maior comprimento do nó até uma folha.
    \item A altura de uma árvore é a altura de sua raiz.
    \item Altura da árvore é a maior profundidade de qualquer nó na árvore.
\end{itemize}
\newpage
\section{Resolução dos Problemas}
Depois de definir o que é indução matemática e indução estrutural, vamos resolver dois problemas que foi proposto. Nessas duas questões a ser resolvidas, vamos usar os conceitos que estabelecemos na primeira parte desse documento para nos ajudar na solução. Cada passo irá ser detalhadamente exposto nesse trabalho, para o leitor ter um bom entendimento de como foi resolvida cada questão.
\hrule
\begin{quote}
        \textbf{PROBLEMA 1}
        Prove a equivalência entre os princípios da indução forte (PIF) e da indução matemática (PIM).
        \newline
        \textbf{PARTE 1}.
        
        
        \textbf{Isto quer dizer que PIF $\leftrightarrow$ PIM.}
       
       
        Vamos começar a resolução desse problema, provando um lado da implicação, que é \textbf{PIF $\to$ PIM}. De uma maneira intuitiva, podemos perceber que provar esse lado da implicação é algo relativamente mais simples, porque se assumimos que a indução forte vale, então o princípio da indução fraca é verdadeira.
        \begin{itemize}
            \item Começamos a resolução dessa parte da implicação, notando que o primeiro passo da indução forte e da fraca são iguais, o que difere uma da outra, e a partir do passo indutivo. Na segunda etapa da indução forte, temos que: P(1) $\wedge$ P(2) $\wedge$.....$\wedge$ P(n) $\to$ P(n + 1). Relembrando o segundo passo da indução fraca, temos algo mais simples P(n) $\to$ P( n +1).
            
            \item Assumindo que o passo base da indução forte é verdadeiro, então consequentemente o passo base da indução fraca também é verídico, porque podemos consumar P( n + 1) a partir de P(a) para todo \textbf{A} entre 1 $>=$ a $<=$ n. No passo indutivo da indução forte, temos que provar que P(1) $\wedge$ P(2) $\wedge$.....$\wedge$ P(n) $\to$ P(n + 1) mas P(1) $\wedge$ P(2) $\wedge$.....$\wedge$ P(n) $\to$ P(n) e, da afirmação do passo indutivo da indução fraca temos que  P(n) $\to$ P( n +1), logo começamos a deixar a questão mais interessante, pois percebemos uma equivalência entre as duas induções. 
            \item De uma forma geral, acabamos de provar a equivalência entre PIF e PIM, pois o primeiro passo de ambas são iguais, então se P(n) vale em PIF implica que P(n) é verdadeiro para PIM, o segundo passo segue o raciocínio do item que explicamos acima. 
        \end{itemize}
        \hrule
        \textbf{PARTE 2}
            Nessa parte, vamos tentar provar o outro lado da implicação, \textbf{PIM} $\to$ \textbf{PIF}.
            \begin{itemize}
                \item Qualquer que seja a propriedade P(n), definida sobre todos os naturais, tem-se:
               \newline
                   (1). (P(1) $\wedge$ $\forall n  $ (P(n) $\to$ P(n + 1))$\to$ $\forall n $ P(n). 
                   
                   Então qualquer propriedade definida sobre os naturais, tem que conter isso:
                   
                   (2). (P(1) $\wedge$ $\forall n  $ (P(1)$\wedge$....$\wedge$P(n) $\to$ P(n + 1))$\to$ $\forall n $ P(n). 
                   
                 
                   Queremos provar que $\forall n$ P(n).         
                   \newline
                   Seja Q(n) a seguinte propriedade (lembrando que a mesma está definida para o conjunto dos números naturais).
                  \newline
                   Q(n) = $\forall k $ (k $\leq$ n $\to$ P(k) $\to$ P(n + 1)) com K $\in$ $\mathbb{N}$.
                   \item Verifica-se que Q(1) (Pois Q(1) é equivalente a P(1) e consequentemente verifica-se P(1)).
                   \item Vamos verificar que $\forall n $ (Q(n) $\to$ Q(n + 1)).
                   Seja n $\in$ $\mathbb{N}$.
                   \newline
                   Se P(1) $\wedge$.....$\wedge$ P(n) $\to$ P(n + 1), então P(1) $\wedge$.....$\wedge$ P(n) $\to$ P(1)$\wedge$.....$\wedge$ P(n) $\wedge$ P(n + 1).
                   
                   Logo, por hipótese $\forall n  $ (P(1) $\wedge$....$\wedge$ P(n) $\to$ P(n + 1)), tem-se que:
                   
                   $\forall n $ (Q(n) $\to$ Q(n + 1)).
                   
                   Por hipótese (2) verifica-se para qualquer propriedade definida sobre os naturais, e portanto também se verifica para Q(n), podemos concluir que:
                   
                   $\forall n $ Q(n).
                   \item Qualquer que seja o natural \textbf{n} Q(n) $\to$ P(n), pois Q(n) = P(1) $\wedge$....$\wedge$ P(n) $\to$ P(n + 1).
                   
                   Observação:
                   
                   Se fizermos uma análise de uma maneira bem cautelosa podemos constatar o passo 2 da PIM e equivalente ao passo 2 da PIF:
                   \newline
                   Passo 2 do PIM: Qualquer que seja o natural n, se P(n) é verdadeiro, então P(n + 1) também é verdadeira.
                   \newline
                   Passo 2 do PIF: Qualquer que seja o natural n, se P(1)$\wedge$....$\wedge$ P(n) são verdadeiras, então P(n + 1) também é verdadeira.
         
            \end{itemize}
    \end{quote}
    \hrule
  
    \begin{quote}
     
        \textbf{PROBLEMA 2}
        Agora utilizaremos o conhecimento adquirido sobre indução para provar a correção de um algoritmo de ordenação de listas conhecido como insertion sort, ou ordenação por inserção.
O pseudocódigo deste algoritmo é dado a seguir:
\begin{quote}
    InsertionSort(l) = $\lbrace l$,               se l =[] 
                    
                    
    
    
\end{quote}
    \end{quote}

    







\section{Conclusão}

\end{document}
